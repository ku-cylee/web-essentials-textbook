\section{Basics of JS Exercise Answers}\label{sect:js-basics-answers}

\subsection*{\excref{exc:validity-of-number}}

\begin{code}{code:validity-of-number-answer}{\excref{exc:validity-of-number} answer}
\begin{minted}{js}
const isValidNumber = num => {
    const parsedNumber = parseInt(num);
    if (!isFinite(parsedNumber) || isNaN(parsedNumber)) return false;
    if (parsedNumber != num) return false;
    return parsedNumber >= 1 && parsedNumber <= 9;
};
\end{minted}
\end{code}

\subsection*{\excref{exc:divisors}}

\begin{code}{code:divisors-answer}{\excref{exc:divisors} answer}
\begin{minted}{js}
const getDivisors = num => {
    const divisors = [];
    for (let i = 1; i <= Math.sqrt(num); i++) {
        if (i * i === num) divisors.push(i);
        else if (num % i === 0) divisors.push(i, num / i);
    }
    return divisors.sort((first, second) => first - second);
};
\end{minted}
\end{code}

\subsection*{\excref{exc:ellipse-object}}

\begin{code}{code:ellipse-object-answer}{\excref{exc:ellipse-object} answer}
\begin{minted}{js}
const ellipse = {
    width: 10,
    height: 5,
    getArea() {
        return Math.PI * this.width * this.height;
    },
    getPerimeter() {
        return 2 * Math.PI * Math.sqrt((this.height ** 2 + this.width ** 2) / 2);
    },
    getEccentricity() {
        return Math.sqrt(1 - (this.height / this.width) ** 2);
    },
};
\end{minted}
\end{code}
