\section{User and Privileges}\label{sect:user-and-privileges}

\subsection*{Database User}

DBMS에는 여러 DB를 생성할 수 있고, 하나의 애플리케이션은 하나의 DB를 사용하는 것이 원칙이므로 여러 애플리케이션이 하나의 DBMS에 접속할 수 있다. root 계정은 소위 관리자 계정에 해당하는 계정으로, DBMS 전체를 조작할 수 있고 모든 DB에 접근하여 데이터를 조작할 수 있어 안전하게 보관되어야 한다.

지금까지는 \sectref{sect:maria-db}의 \coderef{shell:mariadb-login}\과 같이 root 계정을 통해 DB에 접속하였다. 그러나 여러 애플리케이션에서 각자의 DB를 사용하기 위해서 모두 root 계정으로 접속하면 다른 애플리케이션이 사용하는 DB의 데이터를 수정하거나 DB 자체를 없애버리는 등의 조작이 일어날 수 있어 보안상의 문제가 발생할 수 있다. 따라서 관리자는 애플리케이션마다 적어도 하나의 사용자 계정을 생성하고, 각 사용자의 권한을 제한하여 DB 전체가 공격받는 일이 없도록 해야한다.

이번 장에서는 사용자 계정을 생성하고, 사용자에게 특정 DB의 접근 권한을 부여하는 실습을 진행한다.

\begin{sqlenv}{sql:create-user}{Create User}\begin{verbatim}
CREATE USER '<username>'@'<host>' IDENTIFIED BY '<password>'
\end{verbatim}
\end{sqlenv}

먼저 root 계정으로 DB에 로그인한 후, \sqlref{sql:create-user}\과 같이 사용자 계정을 생성한다. 이때 \cd{host}는 사용자가 접속하고자 하는 host로, 생성된 계정은 host에 해당하는 컴퓨터에서만 접속할 수 있다. \cd{host} 값을 \cd{\%}로 설정하면 모든 host에서 접속할 수 있다.

\begin{shellenv}{shell:create-kwebuser-user}{Create \cd{kwebuser} User}\begin{verbatim}
CREATE USER 'kwebuser'@'%' IDENTIFIED BY 'kwebpw';
\end{verbatim}
\end{shellenv}

\shellref{shell:create-kwebuser-user}\는 모든 host로부터 접근할 수 있고, 비밀번호는 ``kwebpw''인 ``kwebuser'' 계정을 생성한 예시이다.

\begin{shellenv}{shell:create-two-dbs}{Create Two DBs}\begin{verbatim}
> CREATE DATABASE kwebdb1;
> CREATE DATABASE kwebdb2;
\end{verbatim}
\end{shellenv}

\shellref{shell:create-two-dbs}\와 같이 두 DB를 생성하고, ``kwebuser'' 사용자에게 \cd{kwebdb1}의 권한만 부여해보자.

\begin{sqlenv}{sql:grant-all-privileges}{Grant All Privileges to User}\begin{verbatim}
GRANT ALL PRIVILEGES ON <db-name>.* TO '<username>'@'<host>'
\end{verbatim}
\end{sqlenv}

\sqlref{sql:grant-all-privileges}\는 사용자에게 \cd{db-name} DB의 모든 권한을 부여하는 SQL문이다. 모든 권한 외에도 옵션에 따라 특정 권한만 부여할 수 있다.

\begin{shellenv}{shell:grant-kwebdb1-to-kwebuser}{Grant All Privileges of \cd{kwebdb1} Table to ``kwebuser'' User}\begin{verbatim}
> GRANT ALL PRIVILEGES ON kwebdb1.* TO 'kwebuser'@'%';
\end{verbatim}
\end{shellenv}

\shellref{shell:grant-kwebdb1-to-kwebuser}\와 같이 \cd{kwebuser} 사용자에게 \cd{kwebdb1} DB의 권한만 부여한다. 이제 root 계정에서 로그아웃하고 \shellref{shell:mariadb-login}\을 참고하여 \cd{kwebuser} 사용자로 로그인해본다.

\begin{shellenv}{shell:kwebuser-access-db}{Accessing DB with ``kwebuser'' User}\begin{verbatim}
$ mysql -ukwebuser -pkwebpw
> USE kwebdb1;
> USE kwebdb2;
\end{verbatim}
\end{shellenv}

\shellref{shell:kwebuser-access-db}\와 같이 ``kwebuser'' 사용자로 로그인한 후, \cd{kwebdb1} DB와 \cd{kwebdb2} DB에 각각 접속을 시도해보면 \cd{kwebdb1}에는 접속할 수 있으나 \cd{kwebdb2}에는 접속이 불가능하다는 것을 확인할 수 있다.
