\section{Introduction to Database}\label{sect:intro-to-database}

\subsection*{데이터베이스의 필요성}

프로그램은 기본적으로 프로그램 내에서 필요한 데이터의 값을 메모리에 저장합니다. 메모리에 저장된 데이터는 access time이 매우 빠르지만, 프로그램이 종료되면 데이터가 사라지면서 더 이상 사용할 수 없게 됩니다. 이러한 특성 때문에 메모리는 휘발성(volatile) 기억장치라고 불립니다.

그러나 대부분의 웹 애플리케이션은 프로그램이 종료되더라도 지워지면 안 되는 데이터를 다룹니다. 예를 들어, 웹 사이트를 이용하는 사용자의 아이디나 비밀번호, 사용자가 작성한 게시글 등이 서버가 종료되었다고 해서 유실되면 매우 곤란한 상황이 발생합니다. 특히 금융기관의 서버에서 고객들의 송금 내역 등의 데이터나 온라인 쇼핑몰 서버에서 고객들의 주문 내역 등이 유실되면 금전적인 피해로 이어질 수 있습니다. 그러므로 웹 애플리케이션은 프로그램이 종료되더라도 데이터가 유실되지 않는, 비휘발성 기억장치에 데이터를 저장해야 합니다.

대표적인 비휘발성(un-volatile) 기억장치는 디스크입니다. 많은 프로그램은 디스크에 파일을 쓰는 작업을 수행하여 프로그램이 종료되더라도 데이터가 유실되지 않도록 저장합니다. 웹 애플리케이션에서도 데이터를 파일로 저장하여 보관할 수 있으나, 데이터를 텍스트(plain text) 파일로 데이터를 저장하는 것은 매우 많은 데이터를 다루기 어렵다는 점, 여러 사용자가 동시에 수정할 수 없다는 점 등의 수많은 문제점이 있습니다.

이러한 문제점 때문에, 수많은 애플리케이션은 매우 많은 정형화된 데이터를 저장하기 위해 텍스트 파일 대신 데이터베이스를 사용합니다. 데이터베이스는 사용자 프로그램이 종료되더라도 데이터를 안전하게 보관하며, 수많은 데이터를 빠르게 탐색할 수 있는 기능을 제공합니다. 또한, 여러 사용자가 데이터베이스의 데이터를 쓰거나 수정할 수 있어 여러 애플리케이션에서 데이터베이스의 데이터에 접근할 수 있으며, 이렇게 바뀐 데이터는 데이터베이스에 거의 실시간으로 반영되어 사용될 수 있습니다.

\subsection*{CRUD}

CRUD는 Create(생성), Read(조회), Update(수정), Delete(삭제) 등 데이터를 다루는 프로그램이 기본적으로 갖추고 있어야 하는 데이터 처리 기능을 묶어 부르는 용어입니다. 각 기능을 자세히 설명하자면, 생성 기능은 데이터를 새로 생성하는 기능, 조회 기능은 저장된 데이터를 조회하여 읽는 기능, 수정 기능은 이미 저장된 데이터를 수정하는 기능, 삭제 기능은 저장된 데이터를 삭제하는 기능을 뜻합니다. 데이터베이스는 사용자가 데이터를 손쉽게 다룰 수 있도록 이러한 CRUD 기능을 제공합니다.

여담으로, HTTP의 메서드도 CRUD 기능에 맞게 설계되었습니다. GET 메서드는 Read(조회), POST 메서드는 Create(생성), PUT 메서드는 Update(수정), DELETE 메서드는 Delete(삭제) 기능에 해당합니다.

\subsection*{Relational DB}

Relational DB, 즉 관계형 데이터베이스(이하 RDB) 모델은 체계화된 DB가 처음 개발된 이래 오늘날까지도 가장 널리 쓰이고 있는 DB 모델입니다. RDB는 데이터를 key와 value들의 간단한 관계를 테이블의 형태로 나타내고, 각 데이터를 column과 row의 형태로 관리하는 모델입니다. Row는 각각의 데이터를 나타내며, record라고 불리기도 합니다. Column은 각 데이터를 구성하는 속성값으로, attribute나 field라고 불리기도 하며, 각 데이터의 이름은 column 이름이라고 불립니다.

\figures{fig:rdb-model-table-example}{Example of RDB Model Table}{
    \fig{images/database-designing/rdb-model-table-example.PNG}{.6}
}

\figref{fig:rdb-model-table-example}\은 컴퓨터학과의 전공과목 목록의 일부를 RDB 모델로 모델링한 것입니다. 각 전공과목의 정보를 row 단위로 저장하고, 각 row의 이수구분, 학수번호, 교과목명, 학점 등의 정보를 각 column에 저장하여 모든 row의 형태가 일관되는 것을 확인할 수 있습니다. 이러한 row와 column의 집합을 table이라고 하고, 하나의 데이터베이스는 다수의 table을 가질 수 있습니다.

이처럼 RDB 모델은 데이터를 매우 직관적이고 일관성 있게 저장하고, 이로 인해 데이터를 분류, 정렬, 탐색하는 속도가 매우 빠르다는 장점이 있습니다. 이러한 장점에 힘입어 대부분의 상용 데이터베이스는 RDB 모델로 설계되어 있고, 대부분의 웹 애플리케이션의 서버에서 RDB 모델을 사용하여 데이터를 저장합니다. 이렇듯 RDB 모델이 가장 기본적이고 대중적인 데이터베이스 모델이기 때문에, 본 스터디에서는 RDB에 대해 스터디할 것입니다.

\subsection*{SQL}

RDB 모델로 설계된 데이터베이스에 저장된 데이터를 다루기 위해 SQL(Structured Query Language)이라는 언어를 사용합니다. SQL은 대부분의 관계형 데이터베이스의 데이터를 다루는 데에 사용되며, SQL을 이용하여 데이터에 대해 CRUD 기능을 수행하거나, 테이블을 생성하는 등의 작업을 수행할 수 있습니다. 준회원 스터디에서는 SQL을 이용하여 RDB형 데이터베이스에 저장된 데이터를 다루고, 더 나아가 이를 Express.js 애플리케이션과 연동하는 스터디를 진행할 것입니다.
