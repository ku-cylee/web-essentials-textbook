\section{MariaDB}\label{sect:maria-db}

\subsection*{DBMS}

흔히 DB 프로그램이라고 칭하는 것은 데이터베이스 관리 시스템(Database Management System; DBMS)입니다. DBMS는 데이터베이스의 데이터를 다루는 역할뿐만 아니라 데이터베이스 생성, 사용자 생성, 사용자의 권한 설정 등 데이터베이스 전반을 관리하는 역할을 합니다.

대표적인 RDB형 DBMS에는 MySQL, Oracle, MSSQL, PostgreSQL 등이 있습니다. 이 중 MySQL은 오픈 소스 소프트웨어라는 점, 표준에 가까운 SQL을 사용한다는 점 때문에 오픈 소스 DBMS 중 가장 높은 점유율을 자랑하였습니다. 그러나 MySQL이 Oracle에 인수된 이후, 라이선스 상태가 불안정하게 되자 MySQL의 개발팀이 MySQL의 코드를 기반으로 MariaDB라는 DBMS를 개발하였습니다. MariaDB에서 MySQL의 기능을 완벽히 사용할 수 있으며, 최근에는 MySQL의 성능과 기능을 추월하여 MariaDB를 사용하는 사례가 급증하는 추세입니다.

준회원 스터디에서는 MariaDB 10.5.6 버전을 이용하여 스터디를 진행하며, 이번 장에서는 MariaDB 서버 프로그램을 다운로드하여 설치할 것입니다.

\subsection*{Installation (Windows)}

https://downloads.mariadb.org/에서 자신의 OS와 버전에 맞는 MSI 패키지를 다운로드하고, 실행합니다.

\figures{fig:mariadb-installation-windows}{Installation of MariaDB server on Windows}{
    \fig{images/database-designing/mariadb-installation-windows.png}{.9}
}

Figure 4.2a와 같이 새로운 DB 인스턴스를 생성하고, Figure 4.2b처럼 root 계정의 비밀번호를 설정해줍니다. root 계정은 DB의 모든 데이터를 다룰 수 있는 관리자 권한을 가지고 있고, root 비밀번호를 잊어버리면 DBMS를 재설치하는 것 외에는 접속 권한을 가질 수 없으므로 신중하게 정하고 기억해야 합니다.

한글 및 기타 특수문자를 사용하기 위해 UTF8을 서버의 기본 character set으로 설정해주고, Figure 4.2c와 같이 TCP port에 MariaDB에 접속하기 위한 port 번호를 설정해야 합니다. 포트 번호는 0-65535 사이의 정수로 설정할 수 있으며, MariaDB는 3306번이 기본 port 번호입니다. MySQL이 이미 설치되어있는 등의 이유로 3306번 port가 선점되어 있다면 3307이나 다른 포트로 변경하여도 됩니다.

\figures{fig:mariadb-env-variable}{Setting Environment Variable}{
    \fig{images/database-designing/mariadb-env-variable.png}{.9}
}

이제 cmd를 이용하여 MariaDB에 접속할 수 있도록 환경 변수를 설정합니다. 먼저 윈도우 탐색기를 열어, 설치 폴더인 C:\textbackslash{}Program Files\textbackslash{}MariaDB 10.5\textbackslash{}bin 폴더로 이동하여, mysql.exe 파일이 있는 것을 확인한 뒤 폴더의 경로를 복사합니다. 윈도우 검색을 이용하여 [시스템 환경 변수 편집]을 실행한 후, \figref{fig:mariadb-env-variable}\과 같이 [환경 변수] $\rightarrow$ [시스템 변수] $\rightarrow$ Path $\rightarrow$ [편집] $\rightarrow$ [새로 만들기]를 선택한 후, 앞에서 복사한 경로를 붙여넣습니다.

\begin{shellenv}{shell:mariadb-check-install-windows}{Checking MariaDB Installation (Windows)}\begin{verbatim}
$ mysql --version
\end{verbatim}
\end{shellenv}

이제 cmd를 이용하여 MariaDB에 접속할 수 있습니다. cmd 창을 열고, \shellref{shell:mariadb-check-install-windows}\와 같이 MariaDB가 제대로 설치되었는지 확인합니다.

\subsection*{Installation (macOS)}

\begin{shellenv}{shell:install-homebrew}{Installing Homebrew}\begin{verbatim}
$ xcode-select --install
$ ruby -e "$(curl -fsSL https://raw.githubusercontent.com/Homebrew/install/master/install)"
$ brew doctor
$ brew update
\end{verbatim}
\end{shellenv}

macOS는 먼저 \shellref{shell:install-homebrew}를 따라 패키지 관리자인 Homebrew를 설치하고, 설치되었는지 확인합니다.

\begin{shellenv}{shell:mariadb-installation-macos}{Installing MariaDB (macOS)}\begin{verbatim}
$ brew info mariadb
$ brew install mariadb
$ brew services start mariadb
\end{verbatim}
\end{shellenv}

Homebrew를 이용해 \shellref{shell:mariadb-installation-macos}\를 참고하여 MariaDB를 설치하고, 실행합니다. 자세한 설치 방법은 MariaDB에서 제공하는 설치 가이드\footnote{https://mariadb.com/resources/blog/installing-mariadb-10-1-16-on-mac-os-x-with-homebrew/}를 참고하여 설치합니다.

\subsection*{Installation (Ubuntu)}

\begin{shellenv}{shell:mariadb-installation-ubuntu}{Installing MariaDB (Ubuntu)}\begin{verbatim}
$ sudo apt-get install software-properties-common
$ sudo apt-key adv --recv-keys --keyserver hkp://keyserver.ubuntu.com:80 0xF1656F24C74CD1D8
$ sudo add-apt-repository 'deb [arch=amd64,arm64,ppc64el] http://mariadb.mirror.liquidtelecom.com/repo/10.5/ubuntu bionic main'
$ sudo apt install mariadb-server
\end{verbatim}
\end{shellenv}

Linux의 일종인 Ubuntu OS의 경우 설치 과정이 다소 복잡합니다. 먼저 \shellref{shell:mariadb-installation-ubuntu}\와 같이 MariaDB 웹 사이트에서 repository를 다운로드받고, apt를 통해 설치합니다.

\begin{shellenv}{shell:mariadb-config-ubuntu}{Configuration of MariaDB (Ubuntu)}\begin{verbatim}
$ sudo mysql_secure_installation
Enter current password for root (enter for none): // your root password here
OK, successfully used password, moving on...
Switch to unix_socket authentication [Y/n] n
Change the root password? [Y/n] n
Remove anonymous users? [Y/n] y
Disallow root login remotely? [Y/n] n
Remove test database and access to it? [Y/n] y
Reload privilege tables now? [Y/n] y
Thanks for using MariaDB!
\end{verbatim}
\end{shellenv}

여기까지 완료되었다면 MariaDB 설치가 완료되었습니다.
