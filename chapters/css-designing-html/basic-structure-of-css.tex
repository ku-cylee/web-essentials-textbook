\section{Basic Structure of CSS} \label{sect:basic-structure-of-css}

\subsection*{CSS의 기본적인 구조}

HTML과 유사하게, CSS 역시 스타일이나 레이아웃을 체계적으로 서술하는 컴퓨터 언어이기 때문에 작성할 때 지켜야 하는 특정한 규칙이 있습니다.

\figcmd{fig:css-basic-struct}{Basic Structure of CSS}
    {images/css-designing-html/css-basic-struct.png}{.6}

CSS의 구조는 \figref{fig:css-basic-struct}과 같이 표현될 수 있습니다. 선택자(selector)는 HTML 요소를 태그 이름, 클래스 이름, 아이디, 상태, 속성 등을 기준으로 선택하는 방법을 서술하는 문자열로, 작성 방법은 \sectref{sect:selectors}에서 구체적으로 다룰 것입니다. 선택자에 의해 선택된 요소들에 일괄적으로 적용할 스타일을 중괄호(\verb|{}|) 내부에 서술합니다.

스타일은 속성과 속성값의 집합으로 표현합니다. 속성(property)은 HTML 요소에 적용하고자 하는 디자인 요소로, 너비, 높이, 글자의 색, 폰트의 크기 및 굵기, 테두리 등 250가지가 넘는 매우 다양한 속성들이 존재합니다. 각 속성에는 그에 대응하는 속성값(value)을 지정할 수 있습니다. 예를 들어, 글자의 색상(\verb|color|) 속성에 사용될 수 있는 속성값으로는 \verb|red|, \verb|blue|, \verb|white| 등의 색상 이름이나, \verb|#1ADB9E|와 같이 색상을 hex code로 표현한 값들이 가능합니다. 각 속성과 그에 대응하는 속성값을 \verb|property: value|의 형태로 쓰고, 각 property-value pair를 세미콜론(\verb|;|)으로 구분하여 나열합니다.

당연하게도, 하나의 선택자에만 스타일을 적용할 수 있는 것은 아닙니다. 각 선택자에 대해 \figref{fig:css-basic-struct}과 같이 CSS를 작성한 후 나열하면 해당하는 요소에 CSS 스타일이 적용됩니다. 이렇게 스타일과 레이아웃을 서술했을 때, 하나의 요소에 적용되는 CSS 속성 중 한 속성에 둘 이상의 값이 적용되는 경우가 발생하면, CSS 문서 내에서 뒤쪽 순서에 있는 스타일이 적용됩니다. 

\begin{codeenv}{fig:css-simple-example}{Simple Example of CSS}\begin{verbatim}


.main-panel {
    width: 800px;
    height: 450px;
    border: 1px solid black;
}

.ball {
    width: 80px;
    height: 80px;
    border: 1px solid red;
    border-radius: 40px;
    position: absolute;
}
\end{verbatim}
\end{codeenv}

\subsection*{Application of CSS on HTML}
CSS에 대해 깊이 있게 다루기 전에, HTML 문서에 CSS 코드를 적용해봅시다. CSS 코드를 적용하는 방법은 세 가지가 있습니다. 

첫 번째 방법은 inline style입니다. HTML 요소에 \verb|style| 속성의 값으로 property-value pair를 직접 열거하는 방법입니다. HTML 요소에 개별적으로 스타일을 적용하는 방식이기 때문에 선택자를 쓰지 않습니다. 이 방법은 각 요소가 어떠한 디자인을 가지는지 쉽게 알 수는 있으나, 웹 페이지의 구조를 표현한다는 HTML의 목적에 위배될뿐더러, 각 요소에 스타일을 작성해주어야 하므로 유사한 스타일을 적용하고자 하는 HTML 요소가 많아지면 효율이 매우 떨어지고, 디자인을 수정할 때 수정하고자 하는 요소를 일일이 수정해주어야 하므로 매우 번거롭습니다. 따라서 inline style은 지양되는 스타일이지만, 서식이 있는 텍스트(rich text)를 표현할 때에는 자주 사용됩니다. 

\begin{codeenv}{code:css-app-inline}{Applying CSS with inline style}\begin{verbatim}


<div>
    Already member? <span style="font-weight: bold">Sign In</span>
</div>
<div>
    <span style="color: red">Sign Up</span>
</div>
\end{verbatim}
\end{codeenv}

두 번째 방법은 internal style sheet입니다. \verb|head| 태그 내부에 \verb|<style type="text/css">| 요소를 삽입하고, 그 내부에 CSS 코드를 작성합니다. 선택자를 사용할 수 있으므로 inline style보다는 더 효율적으로 작성할 수 있으나, HTML 본연의 목적에는 여전히 위배되며, 여러 HTML 문서에 동일한 CSS를 적용하는 경우 디자인을 수정하기가 여전히 번거롭습니다. 

\begin{codeenv}{code:css-app-internal}{Applying CSS with internal style sheet}\begin{verbatim}


<style type="text/css">
    .title {
        font-weight: bold;
        font-size: 16px;
    }
    #article-list {
        list-style-type: none;
        font-size: 12px;
    }
</style>

<div class="title">KWEB Study So Far: </div>
<ul id="article-list">
    <li>Ch 1. Introduction to Front-end</li>
    <li>Ch 2. HTML: The Basic Structure</li>
    <li>Ch 3. CSS: Designing HTML</li>
</ul>
\end{verbatim}
\end{codeenv}

마지막 방법은 external style sheet이며, 가장 권장되는 방법입니다. HTML 문서와 CSS 문서를 서로 다른 파일에 작성하고, HTML 문서에서 \verb|head| 태그 내부에 \verb|link| 태그를 이용하여 CSS 파일을 연동합니다. 이때 \verb|link| 태그의 각 속성에 명시해야 하는 값은 다음과 같습니다. 

\begin{itemize}
    \item \verb|type: text/css| - CSS 타입의 문서를 연결합니다
    \item \verb|rel: stylesheet| – 연결하고자 하는 문서가 HTML 문서의 stylesheet라는 것을 명시합니다.
    \item \verb|href| – 연결하고자 하는 CSS 문서의 주소입니다. 
\end{itemize}

먼저 index.html과 style.css를 같은 폴더 내에 생성하고, \coderef{code:css-app-ext-css}와 같이 style.css를 작성합니다. 

\begin{codeenv}{code:css-app-ext-css}{Applying CSS with external style sheet - style.css}\begin{verbatim}


.title {
    font-weight: bold;
    font-size: 16px;
}

#article-list {
    list-style-type: none;
    padding: 0;
}

#article-list > li {
    font-size: 12px;
}
\end{verbatim}
\end{codeenv}

이후, index.html을 \coderef{code:css-app-ext-html}과 같이 작성합니다. \verb|link| 태그의 구조를 확인하면서 작성해봅시다. 

\begin{codeenv}{code:css-app-ext-html}{Applying CSS with external style sheet - index.html}\begin{verbatim}


<head>
    <link type="text/css" rel="stylesheet" href="./style.css">
</head>
<body>
    <div class="title">KWEB Study So Far: </div>
    <ul id="article-list">
        <li>Ch 0. Introduction to Front-end</li>
        <li>Ch 1. HTML: The Basic Structure</li>
        <li>Ch 2. CSS: Designing HTML</li>
    </ul>
</body>
\end{verbatim}
\end{codeenv}

이제 index.html 파일을 웹 브라우저에 열어서 확인해봅시다. CSS 파일에 작성된 디자인이 적용된 것을 확인할 수 있습니다. 이처럼 external style sheet 방식은 웹 페이지의 구조를 표현하고, 스타일을 표현한다는 HTML과 CSS 각각의 목적을 모두 달성하면서도 유지 및 보수가 용이하다는 장점이 있습니다. 
