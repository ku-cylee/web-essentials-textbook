\section{Introducing CSS} \label{sect:introducing-css}

CSS는 Cascading Style Sheet의 약자로, HTML로 작성된 문서가 실제로 웹 브라우저에 어떻게 표현될지 명시해주는 컴퓨터 언어이며, HTML 문서의 지정된 요소에 의도하는 디자인을 적용할 수 있다. 주로 정적인 디자인을 명시하기 위해 사용되나, 각 요소의 상태나 웹 페이지가 표시되는 화면의 크기 등에 따라 동적인 디자인을 명시할 수도 있다.

\subsection*{CSS 등장 배경}
과거 CSS가 존재하지 않았을 때는 HTML 문서에 웹 페이지의 구조뿐만 아니라 디자인 요소까지 작성하였다. 예를 들어, \coderef{code:early-html}과 같이 \texttt{li} 태그는 정보를 저장하는 태그에 지나지 않았고, 텍스트에 스타일을 저장하기 위해 \texttt{font}, \texttt{b} 등의 태그를 사용하여 스타일을 저장하곤 했다. 그러나 시간이 흐르면서 스타일 및 레이아웃에 관한 정보를 훨씬 많이 저장하게 되었고, HTML 문서에는 본래의 목적인 ``문서의 구조 서술''과는 거리가 먼, 디자인과 관련된 부가적인 정보가 지나치게 많이 작성되게 되었다. 이로 인해 HTML은 인간이 읽었을 때에도 문서의 구조를 파악하기 힘들고, 웹 브라우저가 사용자에게 웹 페이지를 렌더링하기 위해 분석하는 작업조차 힘든, 비효율적인 언어가 되었다. 

HTML이 가지는 이러한 비효율적인 면을 개선하기 위해 1996년 CSS가 발표되었고, HTML과 CSS를 분리하면서 HTML에는 가급적 문서에 대한 구조만 서술하고, CSS에는 각 요소에 대한 스타일이나 레이아웃만을 서술하도록 권고되었다. CSS의 도입으로 HTML은 본연의 목적을 되찾아 문서의 구조를 표현하는 효율적인 언어가 되었으며, 더 나아가 웹 브라우저가 여러 웹 페이지에서 공통으로 사용되는 CSS 문서를 서버로부터 이중, 삼중으로 다운로드할 필요가 없어져 웹 페이지 로딩 역시 빨라지게 되었다.

\begin{codeenv}{code:early-html}{Example of Early HTML}\begin{verbatim}


<body>
    <li><font color="red">HTML before CSS existence.</font></li>
    <b>This is a bold text. </b>
    <i>This text is italicized.</i>
</body>
\end{verbatim}
\end{codeenv}
