\section{Selectors} \label{sect:selectors}

\sectref{sect:basic-structure-of-css}에서 CSS의 구조에 관해 다루면서 선택자(selector)를 소개하였다. CSS에서는 선택자에 따라 원하는 HTML 요소에 원하는 스타일을 지정할 수 있다. 이번 절에서는 선택자를 작성하는 방법을 학습해보자. 

\subsection*{Universal Selector}

전체 선택자(universal selector)는 HTML 문서의 모든 요소를 선택하며, \texttt{*}로 표현된다.

\begin{codeenv}{code:universal-selector}{Universal Selector}\begin{verbatim}


* { width: 80% }
\end{verbatim}
\end{codeenv}

\subsection*{Tag, Class, Id Selector}
태그 이름, 클래스 이름, 아이디를 기준으로 요소를 선택하는 선택자이다. 태그 이름은 \texttt{tag-name}, 클래스 이름은 \texttt{.class-name}, 아이디는 \texttt{\#id}의 형태로 작성한다. 또한, 태그 이름, 클래스 이름, 아이디 등으로 표현된 선택자 \texttt{element}로 선택되는 요소 중 클래스 이름이 \texttt{class-name}인 요소를 선택하는 선택자는 \texttt{element.class-name}의 형태로 작성한다.

\begin{codeenv}{code:tag-class-id-selector}{Examples of Tag, Class, Id Selector}\begin{verbatim}


ul { list-style: none }
.title { font-size: 20px }
#article-list { padding: 0 }
.title.recent-article { font-weight: bold }
\end{verbatim}
\end{codeenv}

\subsection*{Child Selector and Descendants Selector}
자식 선택자(child selector)와 자손 선택자(descendants selector)는 두 개 이상의 선택자를 이용하여 요소를 선택하는 선택자입니다. 자식 선택자는 \texttt{parent > child}의 형태로 쓰며, \texttt{parent} 선택자로 선택된 각 요소의 바로 밑에 있는 자식 요소 중 \texttt{child} 선택자를 기준으로 선택한다. 반면, 자손 선택자는 \texttt{parent child}의 형태로 쓰며, \texttt{parent} 선택자로 선택된 요소들의 밑에 있는 모든 자식 요소 중 \texttt{child} 선택자를 기준으로 선택한다.

\begin{codeenv}{code:child-desc-selector}{Understanding Child Selector and Descendants Selector}\begin{verbatim}


<div class="class1"><span id="span1"></span></div>
<div class="class2"><span id="span2"></span></div>
<div class="class1">
    <div><span id="span3"></span></div>
</div>
\end{verbatim}
\end{codeenv}

\coderef{code:child-desc-selector}를 통해 이해해보자. 자식 선택자인 \texttt{.class1 > span}은 \texttt{\#span1}만 선택하나, 자손 선택자인 \texttt{.class1 span}은 \texttt{\#span1}과 \texttt{\#span3}를 선택한다.

\subsection*{Pseudo-class Selector}
가상 클래스 선택자(pseudo-class selector)는 특정한 상태에 놓여있는 요소들을 선택하는 선택자이다. 예를 들어 \texttt{button} 태그로 구현한 버튼에 일반적인 상태, 호버된 상태, 비활성화된 상태 등의 상태를 클래스의 형태로 나타낸 것을 가상 클래스(pseudo-class)라고 하며, 각각 다른 스타일을 적용할 수 있다. 이러한 가상 클래스 선택자는 \texttt{:pseudo-class}의 형태로 쓰고, \texttt{element:pseudo-class}로 표현된 선택자는 \texttt{element} 선택자로 선택된 요소 중 \texttt{pseudo-class}에 해당하는 요소들을 선택한다. \texttt{a} 태그는 \texttt{:link}, \texttt{:visited}, \texttt{:hover}, \texttt{:active} 등의 가상 클래스를 가질 수 있다.

\begin{codeenv}{code:pseudo-class-selector-1}{Example of Pseudo-class Selector (1)}\begin{verbatim}


<style>
    button { color: black }
    button:hover { color: red }
    button:disabled { background-color: yellow }
</style>
<button>Enabled Button</button>
<button disabled>Disabled Button</button>
\end{verbatim}
\end{codeenv}

가상 클래스에는 HTML 요소의 상태뿐만 아니라 구조와 관련된 가상 클래스도 있다.

\begin{itemize}
    \item \texttt{:first-child} – 선택된 요소 중 가장 첫 번째 요소를 선택한다.
    \item \texttt{:last-child} – 선택된 요소 중 가장 마지막 요소를 선택한다.
    \item \texttt{:nth-child(e(n))} – $n$으로 표현된 식 $e(n)$에 대해, 선택된 요소 중 $e(0)$, $e(1)$, $e(2)$, $\cdots$번째 요소를 모두 선택한다. \texttt{e(n)}은 $an+b$의 꼴만 가능하며, \texttt{e(n)} 대신 \texttt{odd}나 \texttt{even}을 사용할 수 있다.
    \item \texttt{:nth-last-child(e(n))} – \texttt{nth-child}와 유사하게 동작하나, 뒤에서부터 선택한다.
\end{itemize}

이 외에도, \texttt{:not(selector)}은 \texttt{selector}에 의해 선택되지 않는 요소들만 선택한다.

\begin{codeenv}{code:pseudo-class-selector-2}{Example of Pseudo-class Selector (2)}\begin{verbatim}


<style>
    li:first-child { color: red }
    li:last-child { color: blue }
    li:nth-child(4n+3) { color: yellow }
    li:nth-last-child(4n+3) { color: green }
    li:not(#current) { font-style: italic }
</style>
<ul>
    <li>01</li>
    <li>02</li>
    <li>03</li>
    <li>04</li>
    <li>05</li>
    <li id="current">06</li>
    <li>07</li>
    <li>08</li>
    <li>09</li>
    <li>10</li>
</ul>
\end{verbatim}
\end{codeenv}

가상 클래스의 종류는 매우 많으므로, 자주 쓰이는 몇 가지를 제외하고는 필요할 때마다 찾아서 사용하면 된다. 더 많은 가상 클래스는 아래의 MDN reference에서 확인할 수 있다. 

\begin{itemize}
    \item https://developer.mozilla.org/en-US/docs/Web/CSS/Pseudo-classes
\end{itemize}

이번 절에서 다양한 선택자에 대하여 알아보았으며, 이 절에서 언급한 선택자 외에도 속성 선택자, 동위 선택자, 가상 요소 선택자 등 다양한 선택자가 존재한다. 교재에서 언급되지 않은 선택자들은 아래 제시된 링크들에서 확인할 수 있다.

\begin{itemize}
    \item https://www.w3schools.com/cssref/css\_selectors.asp
    \item https://developer.mozilla.org/en-US/docs/Web/CSS/CSS\_Selectors
    \item https://css-tricks.com/almanac/selectors
\end{itemize}
