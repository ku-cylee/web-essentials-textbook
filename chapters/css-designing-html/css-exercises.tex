\section{CSS Exercises} \label{sect:css-exercises}

\subsection*{Exercise 1}
\coderef{code:exercise-responsive-web-1}를 참고하여, \texttt{body} 태그의 배경색을 800px 미만일 때는 \texttt{skyblue}, 800px 이상 1200px 미만일 때는 \texttt{blue}, 1200px 이상일 때는 \texttt{darkblue}가 되도록 style.css를 작성하여라. \texttt{h1} 태그의 색은 \texttt{white}로 고정되어야 한다.

\begin{codeenv}{code:exercise-responsive-web-1}{Exercise 1}\begin{verbatim}


<!doctype html>
<html>
<head>
    <title>Title</title>
    <meta charset="utf-8">
    <meta name="viewport" content="width=device-width, initial-scale=1" >
    <link rel="stylesheet" text="type/css" href="./style.css">
</head>
<body>
    <h1 style="color: white">Responsive Web</h1>
</body>
</html>
\end{verbatim}
\end{codeenv}

\subsection*{Exercise 2}
\coderef{code:exercise-responsive-web-2}을 참고하여 웹 페이지의 너비가 800px 이상이면 \figref{fig:exercise-responsive-web-figure}a, 800px 미만이면 \figref{fig:exercise-responsive-web-figure}b와 같이 렌더링 되도록 style.css 파일을 작성하여라.

\figcmd{fig:exercise-responsive-web-figure}
    {Webpage view of Code 2.20 when the screen width is (a) wider than 800px and (b) narrower than 800px.}
    {images/css-designing-html/responsive-layout.png}{.6}

\begin{codeenv}{code:exercise-responsive-web-2}{Exercise 2}\begin{verbatim}


<!doctype html>
<html>
<head>
    <title>Title</title>
    <meta charset="utf-8">
    <meta name="viewport" content="width=device-width, initial-scale=1" >
    <link rel="stylesheet" text="type/css" href="./style.css">
</head>
<body>
    <div class="header"></div>
    <div class="body">
        <div class="navbar"></div>
        <div class="content"></div>
    </div>
    <div class="footer"></div>
</body>
</html>
\end{verbatim}
\end{codeenv}
