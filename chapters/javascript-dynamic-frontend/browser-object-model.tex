\section{Browser Object Model (BOM)} \label{sect:browser-object-model}

브라우저 객체 모델(Browser Object Model; BOM)은 웹 브라우저의 정보에 접근하여 읽고 제어할 수 있도록 하는 객체 모델이다. BOM은 DOM과는 달리 표준이 없으나, 현대의 브라우저들은 서로 같은 BOM 속성과 메서드를 가지고 출시되므로 브라우저별로 다른 BOM 코드를 작성할 필요가 없다.

\subsection*{\texttt{window} 객체}

DOM에서 HTML 문서의 구조와 관련된 모든 객체가 포함된 \texttt{document} 객체와 유사하게, BOM에서 브라우저와 관련된 객체들은 모두 \texttt{window} 객체에 포함되어 있다. 예를 들어, \coderef{code:js-simple-example}에서 보았던 \texttt{alert} 역시 \texttt{window} 객체의 메서드이며, \texttt{document} 객체 역시 \texttt{window} 객체의 하위 객체이다.

\begin{codeenv}{code:bom-example}{BOM Example}\begin{verbatim}
> window.alert('This is alert 1');
> alert('This is alert 2');
\end{verbatim}
\end{codeenv}

\coderef{code:bom-example}\과 같이 \texttt{window} 객체의 속성이나 메서드는 \texttt{window} 객체를 명시하지 않아도 접근하거나 호출할 수 있다. 이러한 이유로 \texttt{document}가 \texttt{window}의 하위 객체임에도 불구하고 \texttt{window.document}로 작성하지 않고 \texttt{document}로 작성하여도 접근할 수 있다.

다음은 \texttt{window} 객체에서 유용하게 사용되는 세 가지 메서드이다.

\begin{itemize}
    \item \texttt{alert([message])}: 주어진 문자열 경고 메시지로 하는 대화 상자를 띄우는 메서드
    \item \texttt{confirm([message]) => boolean}: 주어진 문자열을 확인 메시지로 하는 대화 상자를 띄워, [확인]이 눌리면 \texttt{true}, [취소]가 눌리면 \texttt{false}를 반환하는 메서드; 남용하지 않는 것이 중요하다
    \item \texttt{open(url, windowName[, windowFeature]) => window}: \texttt{url}에 위치한 리소스를 \texttt{windowName}을 이름으로 하는 새로운 창을 열어 새로운 창 객체를 반환하는 메서드
    \item \texttt{close()}: 현재 창을 닫는 메서드
\end{itemize}

\begin{codeenv}{code:window-methods}{\texttt{window} Object Methods}\begin{verbatim}
> confirm('Cancel order?');
> open('https://www.google.com', 'Window Name', 'width=500, height=400');
> close();
\end{verbatim}
\end{codeenv}
\newpage

\subsection*{\texttt{screen} 객체}

\texttt{window.screen} 객체는 현재 웹 브라우저가 위치한 화면의 크기에 관한 객체이다.

\begin{itemize}
    \item \texttt{width}: 화면의 너비를 값으로 갖는 속성
    \item \texttt{height}: 화면의 높이를 값으로 갖는 속성
    \item \texttt{availWidth}: 화면에서 작업 표시줄 등을 제외하고 사용할 수 있는 영역의 너비를 값으로 갖는 속성
    \item \texttt{availHeight}: 화면에서 작업 표시줄 등을 제외하고 사용할 수 있는 영역의 높이를 값으로 갖는 속성
    \item \texttt{colorDepth}: 화면이 나타낼 수 있는 색상의 수를 bit 단위로 나타낸 값을 갖는 속성; 예를 들어 24 bits로 표현 가능한 색상은 $2^{24} \approx 1.68 \times 10^7$가지이다.
\end{itemize}

\subsection*{\texttt{location} 객체}

\texttt{window.location} 객체는 현재 웹 페이지의 주소(위치)에 대해 다루는 객체이며, \texttt{hostname}, \texttt{pathname}, 프로토콜 등은 back-end 스터디의 HTTP 부분에서 다룰 예정이다.

\begin{itemize}
    \item \texttt{hostname}: 현재 페이지 호스트의 도메인 이름을 값으로 갖는 속성
    \item \texttt{href}: 현재 페이지의 주소를 값으로 갖는 속성
    \item \texttt{pathname}: 현재 페이지의 경로를 값으로 갖는 속성
    \item \texttt{protocol}: 현재 페이지의 프로토콜(HTTP, HTTPS 등)을 값으로 갖는 속성
    \item \texttt{port}: 현재 페이지의 포트를 값으로 갖는 속성
    \item \texttt{assign(url)}: 주어진 문자열을 주소로 하는 웹 문서를 새로 여는 메서드
\end{itemize}

\subsection*{\texttt{history} 객체}

\texttt{window.history} 객체는 방문한 웹 페이지의 목록을 다루는 객체이다.

\begin{itemize}
    \item \texttt{back()}: 현재 페이지의 이전 페이지를 여는 메서드
    \item \texttt{forward()}: 현재 페이지 다음 페이지를 여는 메서드.
    \item \texttt{go(delta)}: 현재 페이지를 기준으로 주어진 페이지 수만큼 떨어진 페이지를 여는 메서드; \texttt{delta=2}라면 두 페이지 앞, \texttt{delta=-2}이면 두 페이지 뒤의 페이지를 열며, \texttt{delta=0}이거나 주어지지 않은 경우 현재 페이지를 reload한다.
\end{itemize}

\subsection*{Timeout and Interval}

웹 페이지를 작성할 때 공이 포물선 형태로 날아가거나, 일정 시간마다 데이터를 가져와 갱신하는 기능 등 특정한 시간 간격으로 기능이 수행되게끔 구현해야 할 때가 있다. BOM에서는 이러한 로직을 쉽게 생성하고 다룰 수 있는 메서드를 제공한다.

먼저, \texttt{setTimeout} 메서드는 함수 \texttt{fn}과 시간 \texttt{delay}를 인자로 받아, 이 로직을 제어할 수 있는 값 \texttt{timeoutId}를 반환하고, 메서드 호출 \texttt{delay} 밀리초 후 함수 \texttt{fn}을 실행하는 메서드이다. \texttt{clearTimeout} 메서드는 \texttt{timeoutId}를 인자로 받아 해당 로직을 중단하는 메서드이다.

\begin{codeenv}{code:timeout-methods}{Timeout Methods}\begin{verbatim}
> const timeoutId = setTimeout(() => {
      console.log('setTimeout 3000');
  }, 3000);
> clearTimeout(timeoutId);
\end{verbatim}
\end{codeenv}

또한, \texttt{setInterval} 메서드는 함수 \texttt{fn}과 시간 \texttt{delay}를 인자로 받아, 이 로직을 제어할 수 있는 값 \texttt{intervalId}를 반환하고, 메서드 호출 \texttt{delay} 밀리초 이후부터 함수 \texttt{fn}을 \texttt{delay} 밀리초마다 반복적으로 실행하는 메서드이다. \texttt{clearInterval} 메서드는 \texttt{intervalId}를 인자로 받아 해당 로직을 중단하는 메서드이다.

\begin{codeenv}{code:interval-methods}{Interval Methods}\begin{verbatim}
> const helloLoopId = setInterval(() => {
      console.log('Hello');
  }, 400);
> const worldLoopId = setInterval(() => {
    console.log('World');
  }, 1000);
> clearInterval(helloLoopId);
> clearInterval(worldLoopId);
\end{verbatim}
\end{codeenv}
