\section{Class and Id Attributes} \label{sect:class-and-id-attributes}

이제까지 HTML 코드를 설계하는 방법에 대해 학습하였습니다. 앞으로의 학습에서는 HTML 요소에 CSS와 JS 코드를 적용하는 과정에 대해 학습할 것입니다. CSS에서는 각 요소에 원하는 디자인을 적용할 수 있고, JS에서는 각 요소를 추가 및 삭제하거나, 그 속성을 수정하는 등의 작업을 할 수 있습니다. 이렇게 CSS와 JS를 HTML 문서에 적용할 때, 특정 요소 혹은 특정 분류에 속해 있는 모든 요소에 CSS나 JS 코드를 적용하는 일이 매우 많을 것입니다. 

앞의 \sectref{sect:commonly-used-html-tags}에서 HTML 코드를 구조화하기 위해 <div>, <span> 태그를 사용한다는 것을 학습하였는데, 이 두 태그만으로는 기능이나 역할 등 원하는 분류 기준에 따라 HTML 요소들을 분류하는 것은 어려우며, HTML 코드가 매우 길어진다면 단순히 태그 이름으로만 요소들을 구분하는 것은 불가능합니다. 

이렇게 특정 HTML 요소를 지정하거나, 특정 기준에 따라 요소들을 분류할 때 필요한 속성이 class와 id입니다. Class와 id는 모든 HTML 요소에 적용할 수 있으며, CSS, JS와 HTML을 연동할 때 매우 중요한 역할을 합니다. 

\subsection*{Class Attribute}
먼저, class 속성은 HTML 요소들을 특정한 기준에 따라 분류(classify)할 때 사용되는 속성이며, class 속성의 값을 class name이라고 합니다. 특정한 기준으로 분류하였을 때 하나의 묶음으로 묶이는 요소들에는 각각 같은 이름의 class를 사용합니다. 하나의 HTML 요소는 여러 class를 가질 수 있고, 각 class name은 공백을 이용하여 구분합니다. 

\begin{codeenv}{code:class-attr-ex}{Class Attributes Example}\begin{verbatim}

<div class="page-thumbnail new">
    <img src="/resources/week2_handout.jpg">
    <span class="page-title">
        <a href="/study/201R/3">Week 2 Handout</a>
    </span>
</div>
<div class="page-thumbnail">
    <img src="/resources/week1_asgmt.jpg">
    <span class="page-title">
        <a href="/study/201R/2">Week 1 Assignment</a>
    </span>
</div>
<div class="page-thumbnail">
    <img src="/resources/week1_handout.jpg">
    <span class="page-title">
        <a href="/study/201R/1">Week 1 Assignment</a>
    </span>
</div>
\end{verbatim}
\end{codeenv}

\subsection*{Id Attribute}
Id 속성은 특정한 HTML 요소 하나를 식별(identification)하기 위해 사용되는 속성입니다. 하나의 요소는 여러 id를 가질 수 없고, 특정 id의 값을 갖는 HTML 요소가 여러 개가 되어서는 안 됩니다. 다만, 각 HTML 요소는 class와 id를 동시에 가질 수 있습니다. 

\begin{codeenv}{code:id-attr-ex}{Id Attributes Example}\begin{verbatim}

<div id="article-form">
    <input id="article-title" name="title">
    <textarea id="article-content" name="content"></textarea>
    <button>Submit</button>
</div>
\end{verbatim}
\end{codeenv}

\subsection*{Naming Convention}
Class name이나 id를 작성할 때 반드시 준수해야 하는 작명 규칙(naming convention)은 없으나, 널리 통용되고 권장되는 규칙을 소개해드리고자 합니다.
\begin{itemize}
    \item 대문자의 사용은 지양하고, 소문자로만 구성합니다. 숫자로 시작하지 않습니다. 
    \item 이름은 class나 id에 잘 부합하여 어떠한 기준으로 지어진 이름인지 알기 쉽게 합니다. 
    \item 여러 단어의 조합으로 이루어졌을 때, 하이픈(-)으로 연결합니다. (예: multiple-words)
\end{itemize}
