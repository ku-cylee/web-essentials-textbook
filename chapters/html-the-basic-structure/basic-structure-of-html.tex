\section{Basic Structure of HTML} \label{sect:basic-structure-of-html}

\subsection*{Tags and Elements}
HTML은 HyperText Markup Language의 약자로, 마크업 언어의 일종이다. 마크업 언어란, HTML은 마크업 언어이며, 마크업 언어는 구조를 서술할 때 정해진 마크로 시작하여 마크로 끝나는 언어이다. \coderef{code:html-ex}를 보며 이해해보자.

\begin{codeenv}{code:html-ex}{Example of HTML}\begin{verbatim}


<!doctype html>
<html>
<head>
    <title>Example Domain</title>

    <meta charset="utf-8" >
    <meta http-equiv="Content-type" content="text/html; charset=utf-8" >
    <meta name="viewport" content="width=device-width, initial-scale=1" >
</head>

<body>
    <div>
        <h1>Example Domain</h1>
    </div>
</body>
</html>
\end{verbatim}
\end{codeenv}

\coderef{code:html-ex}을 관찰해보자. Example Domain이라는 내용을 \verb|<title>|과 \verb|</title>|이 감싸고 있는데, 이러한 문자열을 태그(tag)라고 한다. 두 태그에서 \verb|title|을 \textbf{태그 이름(tag name)}이라고 하며, 태그 내부의 내용(Example Domain)을 태그의 \textbf{내용(content)}이라고 한다. 슬래시(\verb|/|)가 붙어있어 뒤쪽에 오는 태그를 끝 태그(end tag)라고 한다. 유사하게, 이 \verb|title| 태그를 \verb|<head>|와 \verb|</head>|가 감싸고 있는 것을 확인할 수 있다. 이처럼 HTML은 기본적으로 내용의 양 끝단을 태그로 감쌈으로써 구조를 표현한다. 다만, \verb|meta| 태그와 같이 end tag가 존재하지 않는, 단일 태그도 존재한다. 이렇게 태그와 그 태그 사이에 wrap 되어있는 내용이 하나의 \textbf{HTML 요소(element)}를 구성한다.\footnote{단일 태그는 그 자체만으로 하나의 요소가 된다.}

\coderef{code:html-ex}을 더 깊게 관찰해보면, \verb|head| 태그의 요소와 \verb|body| 태그의 요소가 \verb|html| 태그의 요소 일부분인 것을 확인할 수 있다. 즉, HTML 요소는 내부에 또 다른 HTML 요소를 포함할 수 있고, 다른 HTML 요소의 일부가 될 수 있으며, 이와 같은 구조를 nested 구조라고 한다. HTML에서는 요소 A가 요소 B의 일부분을 구성할 때, 요소 A를 요소 B의 \textbf{하위 요소(sub element) 혹은 자식 요소(child element)}라고 하고, 요소 B를 요소 A의 \textbf{상위 요소(super element) 혹은 부모 요소(parent element)}라고 한다.

\figcmd{fig:html-elmt-struct}{Structure of HTML Element}
    {images/html-the-basic-structure/html-elmt-struct.png}{.7}

이제까지 살펴본 HTML 요소의 구조를 종합해보면 \figref{fig:html-elmt-struct}와 같이 표현할 수 있다. 태그의 이름은 대소문자를 구별하지 않으나 소문자로 쓰는 것이 원칙이다. 각 요소에는 속성(attribute)을 설정해줄 수 있으며, 이 경우 속성의 이름과 그에 대응하는 값(value)을 태그에 같이 써줍니다. HTML 요소의 속성은 태그나 내용 못지않게 중요한 역할을 하는데, \sectref{sect:commonly-used-html-tags}에서 더 자세히 알아보도록 하자.

\subsection*{HTML의 기본적인 구조}
HTML 문서의 구조에는 가장 기본적인 틀이 정해져 있다. 먼저 \verb|doctype|을 이용하여 문서가 HTML 문서임을 명시히고, 가장 상위 태그로 \verb|html| 태그, 그 아래에 \verb|head|와 \verb|body| 태그가 위치해야 한다. 코드로 나타낸다면 \coderef{code:html-basic-struct}와 같다.\footnote{HTML에서는 주석을 \texttt{<!--}, \texttt{-->}로 표현한다.}

\begin{codeenv}{code:html-basic-struct}{Basic Structure of HTML}
\begin{verbatim}


<!doctype html>
<html>
    <head>
        <!-- Head Element Content -->
    </head>

    <body>
        <!-- Body Element Content -->
    </body>
</html>
\end{verbatim}
\end{codeenv}

\verb|head| 태그는 문서에 대한 전반적인 정보를 담고 있는 태그로, 이 태그에는 \verb|title|, \verb|meta|, \verb|link| 등 다양한 태그들이 사용되나, 가장 대표적인 두 태그만 살펴보자. 먼저, \verb|title| 태그 내부의 텍스트는 웹 페이지의 제목으로, 웹 브라우저를 통해 웹 페이지에 접속했을 때 브라우저의 상단에 표시된다. \verb|meta| 태그는 HTML 문서의 인코딩 방법, viewport, 키워드 등 중요한 정보들을 지정할 수 있는 태그이다. 

\verb|body| 태그는 웹 페이지에서 사용자에게 보여질 부분이 포함되는 태그이다. 개발자가 웹 페이지를 통해 사용자에게 보여주고자 하는 텍스트, 이미지 등의 내용은 \verb|body| 태그 안에 구현되어야 하며, \sectref{sect:commonly-used-html-tags}에서 본격적으로 웹 페이지의 내용을 구현하는 방법에 대한 학습을 진행한다.
