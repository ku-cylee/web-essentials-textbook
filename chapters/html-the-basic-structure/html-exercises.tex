\section{HTML Exercises} \label{sect:html-exercises}

\subsection*{Structurizing HTML Code}
\sectref{sect:basic-structure-of-html}에서 학습한 HTML의 기본 구조, \sectref{sect:commonly-used-html-tags}에서 학습한 <div>, <span> 태그와 \sectref{sect:class-and-id-attributes}에서 학습한 class, id를 이용하여, 앞의 \coderef{code:input-tags}을 HTML 표준에 맞게 수정하고, 자유롭게 구조화해봅시다. (정해진 정답은 없습니다!) \coderef{code:html-exercise-ex}는 앞의 코드를 구조화한 하나의 예시입니다.

\begin{codeenv}{code:html-exercise-ex}{Example of the Exercise}\begin{verbatim}

<!doctype html>
<html>
<head>
    <title>KWEB Registration Form</title>
</head>
<body>
    <div id="register-form">
        <div class="register-row">
            <h3>Input username: </h3>
            <input class="text-input" type="text" name="username">
        </div>

        <div class="register-row">
            <h3>Input password: </h3>
            <input class="text-input" type="password" name="password">
        </div>

        <div class="register-row">
            <h3>Gender: </h3>
            <label><input type="radio" name="gender" value="male">Male</label>
            <label><input type="radio" name="gender" value="female">Female</label>
        </div>

        <div class="register-row">
            <h3>Your Major: </h3>
            <select>
                <option value="cs">Computer Science</option>
                <option value="phy">Physics</option>
                <option value="chm">Chemistry</option>
                <option value="math">Mathematics</option>
            </select>
        </div>

        <div class="register-row">
            <h3>Introduce yourself: </h3>
            <textarea cols="40" rows="5"
                      placeholder="Introduce yourself"
                      name="introduction">
            </textarea>
        </div>
    </div>
</body>
</html>
\end{verbatim}
\end{codeenv}
